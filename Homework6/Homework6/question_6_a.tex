Show that $\lambda_0 = 0$ is an eigenvalue of the problem if and only if $hL = 1$, in which
case the associated eigenfunction is $y_0(x) = x$.

\begin{proof} ($\Longrightarrow$) 
    
    \nl Suppose $\lambda_0 = 0$ is an eigenvalue. Then $X(x) = A + Bx$. By the first boundary condition, $y(0) = A = 0$, so $y(x) = Bx$. Then the second boundary condition, after computing $y'(x) = B$, gives $h y(L) - y'(L) = 0 = h B L - B = B(hL - 1)$. Since $B \neq 0$ to avoid the trivial solution, $hL - 1 = 0$ which implies $hL = 1$.
\end{proof}


\begin{proof}($\Longleftarrow$) 
    
    \nl Suppose $hL = 1$.
\end{proof}