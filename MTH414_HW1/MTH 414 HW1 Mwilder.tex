\documentclass[12pt]{article}
%\usepackage[document]{ragged2e}
\usepackage{array, amssymb, amsthm, linguex, enumerate, amsmath, physics, enumitem, xcolor, graphicx, xparse}
\let\fg\undefined %remove linguex/siunitx naming clash
\usepackage[english]{babel}
\usepackage[letterpaper,top=2cm,bottom=2cm,left=3cm,right=3cm,marginparwidth=1.75cm]{geometry}
\usepackage[colorlinks=true, allcolors=blue]{hyperref}
\usepackage[group-separator={,}]{siunitx} %\num{12345} -> "12,345"
\usepackage{fancyhdr}
\usepackage{notomath}
\usepackage[T1]{fontenc}
\usepackage{multicol}
\usepackage{mathtools}

%Number sets
\newcommand{\R}{\mathbb{R}}
\newcommand{\C}{\mathbb{C}}
\newcommand{\N}{\mathbb{N}}
\newcommand{\F}{\mathbb{F}}
\renewcommand{\Re}{\operatorname{Re}}
\renewcommand{\Im}{\operatorname{Im}}
\renewcommand{\L}[1]{\mathcal{L}\left({#1}\right)} %Linear Map

\newcommand{\pmp}{\,\pm\,} %add small extra space to \pm

\NewDocumentCommand{\ceil}{ s m }{% ceiling brackets
    \IfBooleanTF{#1}%
    {\lceil #2 \rceil}% starred: no-autosizing
    {\left\lceil #2 \right\rceil}% unstarred: autosizing
}

\NewDocumentCommand{\ceiling}{ s m }{% ceiling brackets
    \IfBooleanTF{#1}%
    {\lceil #2 \rceil}% starred: no-autosizing
    {\left\lceil #2 \right\rceil}% unstarred: autosizing
}

\NewDocumentCommand{\floor}{ s m }{% floor brackets
    \IfBooleanTF{#1}%
    {\lfloor #2 \rfloor}% starred: no-autosizing
    {\left\lfloor #2 \right\rfloor}% unstarred: autosizing
}

\NewDocumentCommand{\pars}{ s m }{% parenthesis
    \IfBooleanTF{#1}%
    {( #2 ) }% starred: no-autosizing
    {\left( #2 \right) }% unstarred: autosizing
}

\NewDocumentCommand{\inner}{ s m }{% inner product
    \IfBooleanTF{#1}%
    {\langle #2 \rangle}% starred: no-autosizing
    {\left\langle #2 \right\rangle}% unstarred: autosizing
}

\NewDocumentCommand{\innerconj}{ s m }{% inner product
    \IfBooleanTF{#1}%
    {\overline{\langle #2 \rangle}}% starred: no-autosizing
    {\overline{\left\langle #2 \right\rangle}}% unstarred: autosizing
}

\NewDocumentCommand{\brac}{ s m }{% brackets
    \IfBooleanTF{#1}%
    {[#2] }% starred: no-autosizing
    {\left[ #2 \right] }% unstarred: autosizing
}

%default latex bracket size naming
\newcommand{\biggbrac}[1]{\bigg[ {#1} \bigg] }
\newcommand{\bigbrac}[1]{\big[ {#1} \big] }
\newcommand{\Bigbrac}[1]{\Big[ {#1} \Big] }


\RenewDocumentCommand{\over}{ s m }{% fraction 1/arg
    \IfBooleanTF{#1}%
    {\dfrac{1}{#2}}% starred: dfrac
    {\frac{1}{#2}}% unstarred: normal frac
}

\NewDocumentCommand{\pover}{ s m }{% parenthesis around fraction (1/arg)
    \IfBooleanTF{#1}%
    {\left(\dfrac{1}{#2}\right)}% starred: dfrac
    {\left(\frac{1}{#2}\right)}% unstarred: normal frac
}

\NewDocumentCommand{\pfrac}{ s m m}{% parenthesis around fraction (arg1/arg2)
    \IfBooleanTF{#1}%
    {\left( \dfrac{{#2}}{{#3}} \right)}% starred: dfrac
    {\left( \frac{{#2}}{{#3}} \right)}% unstarred: normal frac
}


\newcommand{\Xbar}{\bar{X}}
\newcommand{\Ybar}{\bar{Y}}
\newcommand{\xbar}{\bar{x}}
\newcommand{\ybar}{\bar{y}}


\newcommand{\limn}{\lim_{n\to\infty}}

\newcommand{\gammaDist}[2]{\operatorname{Gamma} \left( {#1},{#2} \right)} %gamma distribution
\NewDocumentCommand{\normalDist}{s g g}{ %normal distibution
    \IfBooleanTF{#1} { % starred, no autosizing parenthesis
      \IfNoValueTF{#2}{
          N (\mu,\, \sigma^2 ) %\normalDist* "default" normal distribution N(\mu, \sigma^2)
        } {
            \IfNoValueTF{#3}{N (#2)}{} %\normalDist{arg} --> N(arg)
        }
      \IfNoValueTF{#3}{}{N ( #2, #3 )}  %\normalDist*{arg1}{arg2} --> N(arg1,arg2)
    }  % else (unstarred) autosize parenthesis
    {
        \IfNoValueTF{#2}{
            N \left(\mu,\, \sigma^2 \right) %\normalDist "default" normal distribution N(\mu, \sigma^2)
        } {
            \IfNoValueTF{#3}{N \left(#2\right)}{} %\normalDist{arg} --> N(arg)
        }
        \IfNoValueTF{#3}{}{N \left( #2, #3 \right)} %\normalDist{arg1}{arg2} --> N(arg1,arg2)
    }
}



%colors
\definecolor{ggreen}{RGB}{0, 127, 0}
\definecolor{dgray}{RGB}{63,63,63}
\definecolor{neonorange}{RGB}{255,47,0}
\definecolor{mygray}{rgb}{0.5,0.5,0.5}
\definecolor{eblue}{RGB}{0,74,127}
\newcommand{\red}[1]{\color{red}{#1}\color{black}}
\newcommand{\green}[1]{\color{ggreen}{#1}\color{black}}
\newcommand{\blue}[1]{\color{blue}{#1}\color{black}}
\newcommand{\setRed}{\color{red}}
\newcommand{\setBlack}{\color{black}}
\newcommand{\setBlue}{\color{blue}}
\newcommand{\setGreen}{\color{ggreen}}



\newcommand{\thru}[1]{{#1}_1, \dots, {#1}_n}
\newcommand{\sumThru}[1]{{#1}_1 + \cdots + {#1}_n}
\newcommand{\yn}{Y_1, \dots, Y_n} % Y_1, ..., Y_n
\newcommand{\xn}{X_1, \dots, X_n} % Y_1, ..., Y_n

%hats and tildes
\newcommand{\that}{\widehat{\theta}} % theta hat
\newcommand{\phat}{\widehat{p}} % p hat
\newcommand{\qhat}{\widehat{q}} % p hat
\newcommand{\psihat}{\widehat{\psi}} % psi hat
\newcommand{\Psihat}{\widehat{\Psi}} % Psi hat
\newcommand{\ptilde}{\widetilde{p}} % psi tilde
\newcommand{\Psitil}{\widetilde{\Psi}} % Psi tilde
\newcommand{\betah}{\widehat{\beta}} % beta hat

%2x2 matrix shortcuts
\newcommand{\detx}[4]{\begin{vmatrix}{#1} & {#2}\\{#3}&{#4}\end{vmatrix}} % 2x2 determinant
\newcommand{\dety}[9]{\begin{vmatrix}{#1} & {#2} & {#3} \\{#4}&{#5}&{#6}\\ {#7} & {#8} & {#9}\end{vmatrix}} % 3x3 determinant
\newcommand{\bmaty}[9]{\begin{bmatrix}{#1} & {#2} & {#3} \\{#4}&{#5}&{#6}\\ {#7} & {#8} & {#9}\end{bmatrix}} % 3x3 matrix
\newcommand{\bmat}[4]{\begin{bmatrix}{#1} & {#2}\\{#3}&{#4}\end{bmatrix}} % 2x2 matrix brackets
\renewcommand{\pmat}[4]{\begin{pmatrix}{#1} & {#2}\\{#3}&{#4}\end{pmatrix}} % 2x2 matrix parenthesis

%remove any enumerate/itemize indent temporarily
\makeatletter   %% <- make @ usable in macro names
\newcommand*\notab[1]{%
  \begingroup   %% <- limit scope of the following changes
    \par        %% <- start a new paragraph
    \@totalleftmargin=0pt \linewidth=\columnwidth
    %% ^^ let other commands know that the margins have been reset
    \parshape 0
    %% ^^ reset the margins
    #1\par      %% <- insert #1 and end this paragraph
  \endgroup
}
\makeatother    %% <- revert @


\newcommand{\dimrange}[1]{\operatorname{dim}\operatorname{range}{#1}} % dimrange
\newcommand{\dimnull}[1]{\operatorname{dim}\operatorname{null}{#1}} % dimnull
\newcommand{\range}[1]{\operatorname{range}{#1}} %range
\newcommand{\nullspace}{\operatorname{null}} %null

% polynomial notation
\NewDocumentCommand{\poly}{ s g g }{%
    \IfBooleanTF{#1} {
        \IfNoValueTF{#2} {
            \mathcal{P}(\mathbb{R})
        } {
            \mathcal{P}_{#2}(\mathbb{R})
        }
    } {
        \IfNoValueTF{#3} {
            {\mathcal{P}(#2)}
        } { %else
            {\mathcal{P}_{#2}(#3)}
        }
    }
}

\NewDocumentCommand{\bias}{ s m }{% bias(arg)
    \IfBooleanTF{#1}%
    {\operatorname{bias}(#2)}% starred: no autosizing
    {\operatorname{bias}\left(#2\right)}% unstarred: autosizing
}

\NewDocumentCommand{\MSE}{ s m }{% MSE(arg)
    \IfBooleanTF{#1}%
    {\operatorname{MSE}(#2)}% starred: no autosizing
    {\operatorname{MSE}\left(#2\right)}% unstarred: autosizing
}

\NewDocumentCommand{\Var}{ s m }{% variance with parenthesis V(arg)
    \IfBooleanTF{#1}%
    {\operatorname{Var}(#2)}% starred: no autosizing
    {\operatorname{Var}\left(#2\right)}% unstarred: autosizing
}

\NewDocumentCommand{\Varb}{ s m }{% variance with brackets V[arg]
    \IfBooleanTF{#1}%
    {\operatorname{Var}[\,#2\,]}% starred: no autosizing
    {\operatorname{Var}\left[\,#2\,\right]}% unstarred: has autosizing
}

\NewDocumentCommand{\Vb}{ s m }{% another renaming of variance with brackets V[arg]
    \IfBooleanTF{#1}%
    {\operatorname{Var}[\,#2\,]}% starred: no autosizing
    {\operatorname{Var}\left[\,#2\,\right]}% unstarred: has autosizing
}

\NewDocumentCommand{\E}{ s m }{% expectation with parenthesis E(arg)
    \IfBooleanTF{#1}%
    {\operatorname{E}(#2)}% starred: no autosizing
    {\operatorname{E}\left(#2\right)}% unstarred: has autosizing
}

\NewDocumentCommand{\Eb}{ s m }{% expectation with brackets E[arg]
    \IfBooleanTF{#1}%
    {\operatorname{E}[#2]}% starred: no autosizing
    {\operatorname{E}\left[#2\right]}% unstarred: has autosizing
}

\RenewDocumentCommand{\P}{ s m }{% probability with parenthesis Pr(arg)
    \IfBooleanTF{#1}%
    {\Pr (#2) }% starred: no autosizing
    {\Pr \left( #2 \right) }% unstarred: has autosizing
}

\NewDocumentCommand{\prob}{ s m }{% probability with parenthesis Pr(arg)
    \IfBooleanTF{#1}%
    {\Pr (#2) }% starred: no autosizing
    {\Pr \left( #2 \right) }% unstarred: has autosizing
}

\NewDocumentCommand{\eff}{ s m }{% efficiency with parenthesis eff(arg)
    \IfBooleanTF{#1}%
    {\operatorname{eff}(#2)}% starred: no autosizing
    {\operatorname{eff}\left(#2\right)}% unstarred: has autosizing
}

%vertical vector of up to 8 elements
\NewDocumentCommand\vvec{s m g g g g g g g}{%
    \IfBooleanTF{#1} {
        \begin{bmatrix}% if starred use brackets
            \IfNoValueTF{#2}{}{#2}
            \IfNoValueTF{#3}{}{\\#3}
            \IfNoValueTF{#4}{}{\\#4}
            \IfNoValueTF{#5}{}{\\#5}
            \IfNoValueTF{#6}{}{\\#6}
            \IfNoValueTF{#7}{}{\\#7}
            \IfNoValueTF{#8}{}{\\#8}
        \end{bmatrix}
    }  % else (unstarred) use parethesis
    {
        \begin{pmatrix}%
            \IfNoValueTF{#2}{}{#2}
            \IfNoValueTF{#3}{}{\\#3}
            \IfNoValueTF{#4}{}{\\#4}
            \IfNoValueTF{#5}{}{\\#5}
            \IfNoValueTF{#6}{}{\\#6}
            \IfNoValueTF{#7}{}{\\#7}
            \IfNoValueTF{#8}{}{\\#8}
        \end{pmatrix}
    }
}
\def\Cov{\operatorname{Cov}} %Covariance
\def\df{\text{df}} %degrees of freedom

\NewDocumentCommand{\example}{ s g }{% Example header
    \IfBooleanTF{#1}%
    {\vspace{0.1in}}% starred: 0.1in
    {\vspace{0.2in}}% unstarred: 0.2in
    \IfNoValueTF{#2} {\noindent\textbf{\color{eblue} Example: }}{\noindent\textbf{\color{eblue} Example (#2): }}
}
\NewDocumentCommand{\disc}{ s }{% Discussion header
    \IfBooleanTF{#1}%
    {\vspace{0.1in}\noindent\textbf{Discussion: } }% starred: 0.1in
    {\vspace{0.2in}\noindent\textbf{Discussion: } }% unstarred: 0.2in
}
\NewDocumentCommand{\defn}{ s }{% Definition header
    \IfBooleanTF{#1}%
    {\vspace{0.1in}\noindent\textbf{\color{neonorange} Definition: } }% starred: 0.1in
    {\vspace{0.2in}\noindent\textbf{\color{neonorange} Definition: } }% unstarred: 0.2in
}
\NewDocumentCommand{\reason}{ s }{% Reason header
    \IfBooleanTF{#1}%
    {\vspace{0.1in}\noindent\textbf{Reason:} }% starred: 0.1in
    {\vspace{0.2in}\noindent\textbf{Reason:} }% unstarred: 0.2in
}
\NewDocumentCommand{\recall}{ s }{% Recall header
    \IfBooleanTF{#1}%
    {\vspace{0.1in}\noindent\textit{Recall:} }% starred: 0.1in
    {\vspace{0.2in}\noindent\textit{Recall:} }% unstarred: 0.2in
}
\NewDocumentCommand{\remark}{ s }{% Remark header
    \IfBooleanTF{#1}%
    {\vspace{0.1in}\noindent\textit{Remark:} }% starred: 0.1in
    {\vspace{0.2in}\noindent\textit{Remark:} }% unstarred: 0.2in
}

\NewDocumentCommand{\soln}{ s }{% Remark header
    \IfBooleanTF{#1}%
    {\vspace{0.1in}\noindent\textbf{Solution: } }% starred: 0.1in
    {\vspace{0.2in}\noindent\textbf{Solution: } }% unstarred: 0.2in
}

\newcommand{\proj}[2]{\operatorname{proj}_{{#1}}{#2}} %projection
\newcommand{\wideand}{\qquad \text{and} \qquad}

\newcommand{\bu}[1]{\textbf{\underline{{#1}}} } %bold underline
\newcommand{\boldit}[1]{\textbf{\textit{{#1}}} } %bold italix

% put actual quotation marks "around something"
\newcommand{\say}[1]{\textquotedblleft{#1}\textquotedblright}

% max{arg} and min{arg}
\renewcommand{\max}[1]{\operatorname{max}\left\{ #1 \right\}}
\renewcommand{\min}[1]{\operatorname{min}\left\{ #1 \right\}}

\newcommand{\set}[1]{\left\{#1\right\}}

\newcommand{\Span}[1]{\operatorname{span}\left\{ #1 \right\}}

%Create a new vspace line no indent
\newcommand{\nl}{\vspace{0.1in}\noindent}
\newcommand{\nnl}{\vspace{0.2in}\noindent}
\newcommand{\nnnl}{\vspace{0.3in}\noindent}
\textwidth=7.02in
\hoffset=-.425in

\setcounter{MaxMatrixCols}{20}
\begin{document}
\pagestyle{fancy}
\fancyhf{}
\fancyhead[RO]{Matthew Wilder}
\fancyhead[LO]{MTH 414 - Homework \#1}
\fancyfoot[CO]{Page \thepage}

\noindent MTH 414 - Spring 2023
\\Homework \#1 - Calculus/ODE Review
\\Due: Monday, January 23, 2023 (2:00PM)


\begin{enumerate}
\item For each of the following equations, state the order and whether it is nonlinear, linear inhomogeneous, or linear homogeneous; provide reasons;

\begin{enumerate}
\item compute $f_x$ and $f_y$

\soln*

\begin{itemize}
    \item $f_x = y \cos(xy) - 3x^2y + y^4$
    \item $f_y = x \cos(xy) - x^3 + 4xy^3$ 
\end{itemize}

\item compute $f_{xx}$, $f_{xy}$, $f_{yy}$ and $f_{yx}$

\soln*

\begin{itemize}
    \item $f_{xx} = -y^2 \sin(xy) - 6xy$
    \item $f_{yy} = -x^2 \sin(xy) + 12xy^2$
    \item $f_{xy} = f_{yx} = \cos(xy) - xy \sin(xy) -3x^2 + 4y^3$ 
\end{itemize}\vspace{.2in}
\item $\displaystyle \dv{x} \pars{(f\ast g)(x)} = \pars{\dv{f}{x} \ast g}(x)$

\begin{proof}
    \begin{align*}
        \dv{x} \pars{(f\ast g)(x)} &= \dv{x} \pars{\intreal f(x-t)g(t)\dd t} \\
        &=  \intreal \dv{x}f(x-t)g(t)\dd t  \tag{since integrating wrt $t$, not $x$}\\
        &=  \pars{\dv{f}{x} \ast g}(x)
    \end{align*}
\end{proof}
\item $u_{tt} - u_{xx} + \frac{u}{x} = 0$

\soln* Second order linear homogenous since $\mathcal{L} = \pdv{t} - \pdv[2]{x} + \over{x}$ and $g = 0$.\vspace{.2in}
\item $\mcu(e^{-2 \pi i a x } f(x)) = \hat{f}(\gamma - a)$
\begin{proof}
    I changed the exponent sign, I believe it was wrong.
    \begin{align*}
        \hat{f}(\gamma - a) &= \hat{g}(\gamma)\\
        &= \intreal g(t) e^{-2\pi i x t} \dd t \\
        &= \intreal f(t-a)e^{-2\pi i x t} \dd t \\
        &=  \intreal f(u)e^{-2\pi i x (u+a)} \dd u \tag{$u=t-a,\; t = u+a$} \\
        &= e^{-2\pi i x a} \intreal f(u)e^{-2\pi i x u} \dd u \\
        &= e^{-2\pi i x a} \hat{f}(x)
    \end{align*}
\end{proof}\vspace{.2in}
\item $\displaystyle - \dv{\gamma} \hat{f}(\gamma) = \mcu(2\pi i x f(x))$
\begin{proof}
    \begin{align*}
        - \dv{\gamma} \hat{f}(\gamma) &=  - \dv{\gamma}  \pars{ \intreal f(\gamma) e^{-2 \pi i \gamma x} \dd \gamma}\\
        &= \pars{ \intreal f(\gamma) - \dv{\gamma} e^{-2 \pi i \gamma x} \dd \gamma}\\
        &= -\pars{ \intreal f(\gamma) -2 \pi ix e^{-2 \pi i \gamma x} \dd \gamma}\\
        &=  2 \pi ix \pars{ \intreal f(\gamma) e^{-2 \pi i \gamma x}\dd \gamma}\\
        &= \mcu(2\pi i x f(\gamma))
    \end{align*}
\end{proof}\vspace{.2in}
\item $\int_{\R} \hat{f}(x)g(x) = \int_{\R}f(x)\hat{g}(x)$
\begin{proof}
    \begin{align*}
        \intreal  \hat{f}(x)g(x) &= \intreal \pars{\intreal f(t) e^{-2\pi i x t}\dd t}g(x)\dd x\\
        &= \intreal \pars{\intreal f(t) e^{-2\pi i x t}g(x)\dd t}\dd x\\
        &\stackrel{\mathclap{\normalfont\mbox{Fubini}}}{=} \quad \intreal \pars{\intreal f(t) e^{-2\pi i x t}g(x)\dd x}\dd t\\
        &= \intreal \pars{\intreal e^{-2\pi i x t}g(x)\dd x} f(t)\dd t\\
        &= \intreal \hat{g}(t) f(t) \dd t\\
        &= \intreal f(x)\hat{g}(x) \dd x
    \end{align*}
\end{proof}
\end{enumerate}

\item Find the Fourier Transform of the function $f(x) = xe^{-kx^2}$. 

\soln* 

\nnl In order to use the derivitive formulas, first want to compute $\mathcal{F}\pars{e^{-kx^2}}$.

$$\mathcal{F}\pars{e^{-kx^2}} = \intreal e^{-kx^2}e^{-2\pi i t x}\dd x = \intreal e^{-(kx^2 + 2\pi i tx) \dd x}$$
This has a quadratic in $x$, $\;kx^2 + 2\pi i tx$. Completing the square gives 
$$kx^2 + 2\pi i tx + \pfrac{\pi i t}{\sqrt{k}}^2 - \pfrac{\pi i t}{\sqrt{k}}^2 = \pars{\sqrt{k}x + \pfrac{\pi i t}{\sqrt{k}}}^2 - \pfrac{\pi i t}{\sqrt{k}}^2$$

Thus we have 
$$ \intreal \exp{ -\brac{\pars{\sqrt{k}x + \pfrac{\pi i t}{\sqrt{k}}}^2 - \pfrac{\pi i t}{\sqrt{k}}^2}} \dd x = \exp{\pfrac{\pi i t}{\sqrt{k}}^2} \intreal \exp{-\pars{\sqrt{k}x + \pfrac{\pi i t}{\sqrt{k}}}^2}\dd x $$

Letting $u = \sqrt{k}x + \pfrac{\pi i t}{\sqrt{k}}$ gives $\dfrac{1}{\sqrt{k}} \dd u = \dd x$ so 
$$\mathcal{F}\pars{e^{-kx^2}} = \frac{1}{\sqrt{k}} \exp{\pfrac{\pi i t}{\sqrt{k}}^2} \intreal e^{-u^2} \dd u = \frac{1}{\sqrt{k}} \exp{ -\frac{\pi^2 t^2}{k}}  \sqrt{\pi} =  \frac{\sqrt{\pi}}{\sqrt{k}} e^{-\pi^2 t^2 / k}$$

\nnl Next, $\int xe^{-kx^2} \dd x= \pfrac*{e^{-kx^2}}{-2k}$. It follows that 
$$\mathcal{F}\pars{xe^{-kx^2}} = \mathcal{F}\pars{\dv{x}\pfrac{e^{-kx^2}}{-2k}} = 2\pi i t \mathcal{F}\pars{ \over{-2k}\pars{e^{-kx^2}}} =- \frac{\pi i t}{k} \mathcal{F}\pars{ e^{-kx^2}} = \frac{- \pi \sqrt{\pi} i t}{k\sqrt{k}} e^{-\pi^2 t^2 / k}$$
\item (the Laplacian)

\nl Consider the function $f(x,y) = \log(x^2 + y^2)$. Show that $f_{xx} + f_{yy} = 0$ and determine the domain in $\R^2$ for which your calculation is valid 

\soln*

\begin{align*}
    f_{x} &= 2x \Big/ \pars{\ln(10) \cdot (x^2 + y^2) } \\
    f_{xx} &= \pdv{x}\Bigg( \underbrace{2x}_{f} \Big/ \underbrace{  \pars{\ln(10) \cdot (x^2 + y^2) }  }_{g}  \Bigg)
    \\ &= \frac{f'g - fg'}{g^2} \\
    &= \frac{ \pars{2\ln(10)(x^2 + y^2)} - \pars{4x^2 \ln(10)} }{\ln^2(10) (x^2 + y^2)^2  }\\
    &= \frac{ 2(x^2 + y^2) - 4x^2 }{\ln(10) (x^2 + y^2)^2  }\\
    &= \frac{-2x^2 + 2y^2}{\ln(10) (x^2+y^2)^2} \\
    f_{yy} &= \frac{2x^2 - 2y^2}{\ln(10) (x^2+y^2)^2}
\end{align*}
$$f_{xx} + f_{yy} = \frac{(-2x^2 + 2y^2) + (2x^2 - 2y^2)}{\ln(10) (x^2+y^2)^2} = 0.$$
Domain $D := \{(x,y) \mid (x,y) \neq (0,0)\}$
\item Solve the following Laplace's equation on a infinite strip:
$$\begin{cases}
    \Delta u = 0 & 0<x<L, \; y\in\R\\
    u(0,y) = g_1(y)\\
    u(L,y) = g_2(y)
\end{cases}$$

\soln* 

\newpage

$ $
\item (initial value problem)

\begin{enumerate}
\item Given any $\alpha \in \R$, solve the initial value problem
$$y' = y^2 \cos (t),\; y(0) = \alpha.$$

\soln* This is a separable equation.
\begin{align*}
    & y' = \dv{y}{t} = y^2 \cos(t)\\
    \iff & \over{y^2} \dd{y} = \cos(t) \dd{t}\\
    \iff & -\frac{1}{y} = \sin(t) + C\\
    \iff & y(t) = \over{-\sin(t)+C}\\
    \implies &y(0) = \over{C} = \alpha\\
     \iff & C = \over{\alpha}
\end{align*}
Hence the general solution is $y = \dfrac{1}{\over{\alpha}-\sin(t)}$.\vspace{.25in}
\item For what values of $\alpha$ is the solution defined for all time?
(Hint: You may need to treat $\alpha =0$ and $\alpha \neq 0$ separately.)

\soln* The solution is undefined when the denominator, $\over{\alpha} - \sin(t) = 0$. Rearranging this into $\over{\alpha} = \sin(t)$, this has an infinite number of solutions when $\over{\alpha} \in [-1, 1]$. This occurs when $\abs{\alpha} \geq 1$, therefore the solution exists for all time when $\abs{\alpha} < 1$ and $\alpha \neq 0$ (since we cannot divide by zero).

\nl Thus, a solution exists for all $t$ when $\abs{\alpha} < 1 \land \alpha \neq 0$. 
\end{enumerate}
\vspace{.75in}
\item Consider the equation $$u_x + yu_y = 0$$ with the boundary condition $u(x,0) = \phi(x)$.
\begin{enumerate}
\item Find the general solution to the PDE.

\soln* $\pdv{y}{x} = \frac{y}{1} \iff \over{y}\dd{y} = \dd x \iff \ln \abs{y} = x + C \iff y = Ce^x \iff C = e^{-x}y$.

\nl Then $\pdv{x} u(x, Ce^x) = u_x + Ce^x u_y = u_x + yu_y = 0$. Letting $x=0$ then $u(0,Ce^0) = u(0,C) = u(0, e^{-x}y)$. Thus $$u(x,y) = f(e^{-x}y)$$\newpage
\item Show that the problem has a single negative eigenvalue $\lambda_0$ if and only if $hL > 1$, in which case $\lambda_0 = - \beta_0^2 /L^2$ and $y_0 = \sinh \beta_0 x/L$, where $\beta_0$ is the positive root of the equation $\tanh x = x/hL$. (Suggestion: Sketch the graphs of $y = \tanh$ and $y = x/hL$)


\begin{proof} ($\Longrightarrow$) 
    
    \nl  \vspace{3.5in}
\end{proof}


\begin{proof}($\Longleftarrow$) 
    
    \nl \vspace{3.5in}
\end{proof}
\item (BVP without uniqueness) For $\phi(x) \equiv 1$, show that there are many solutions.

\end{enumerate}
\newpage
\item (solution spaces)

\nl Consider the second order linear homogeneous ODE of the form 
$$y'' + P(x)y' + Q(x)y = 0$$
where $P$ and $Q$ are defined on some interval $I$ in $\R$. Show that the set of all solutions to this ODE forms a vector space. That is, verify each of the following:

\begin{enumerate}
\item $y = 0$ is a solution

\begin{proof}
    $y = 0 \implies y'=0 \implies y''= 0$, hence
    \begin{align*}
        y'' + P(x)y' + Q(x)y &= 0 + 0 \cdot P(x) + 0\cdot Q(x) \tag{substitution}\\
        &= 0 + 0 + 0 \tag{properties of zero}
        \\ &= 0 \tag{additive identity on $\R$}
    \end{align*}
    $\therefore \; y=0$ is a solution.  
\end{proof}\vspace{0.25in}
\item Given any two solutions $y_1$ and $y_2$ and scalars $\alpha, \beta \in \R$, the function 
$\alpha y_1 + \beta y_2$ is a solution of the ODE.

\begin{proof}
    Let $f := \alpha y_1 + \beta y_2$, then $f' = \alpha y_1' + \beta y_2'$ and $f'' = \alpha y_1'' + \beta y_2''$.
    Substituting the proposed solution, $f$, into the differential equation gives
    \begin{align*}
        y'' + P(x)y' + Q(x)y &= (\alpha y_1'' + \beta y_2'') + P(x) (\alpha y_1' + \beta y_2') + Q(x) (\alpha y_1 + \beta y_2)\\
        &= \alpha y_1'' + \beta y_2'' + P(x) \alpha y_1' + P(x)\beta y_2' + Q(x) \alpha y_1 + Q(x) \beta y_2\\
        &= \pars{\alpha y_1'' + P(x) \alpha y_1' + Q(x) \alpha y_1} + \pars{
            \beta y_2'' + P(x)\beta y_2' + Q(x) \beta y_2
        }\\
        &= \alpha \pars{ y_1'' + P(x)  y_1' + Q(x)  y_1} + \beta \pars{ y_2'' + P(x) y_2' + Q(x)  y_2}\\
        &= 0\alpha + 0\beta \tag{by hypothesis}\\
        &= 0.
    \end{align*}
    Hence $f := \alpha y_1 + \beta y_2$ is a also solution.
\end{proof}
\end{enumerate}

\vspace{.5in}
In fact, you can show (you don't need to do this here, although it might be good to look up), that a vector space of all solutions to the above ODE has dimension 2 and a basis can be found by finding two linearly independent solutions on the interval $I$.


\nnl
\begin{align*}
    \P{x > 10} &= 1 - \P{x \leq 10}\\
    &= 1 - \sum_{k=0}^{10} \P{x = k} \\
    &= 1 - \sum_{k=0}^{10} {}^nC_k p^k q^{n-k}\\
    &= 1 - \sum_{k=0}^{10} {}^{35}C_k (0.1)^k (0.9)^{35-k}\\
    &= 1 - \sum_{k=0}^{10} \frac{35!}{k!(35-k)!}  (0.1)^k (0.9)^{35-k}\\
    &= 1 - \frac{4997878512022745871397452694244043}{5000000000000000000000000000000000}\\
    &= \frac{2121487977254128602547305755957}{5000000000000000000000000000000000} \\
    &\approx 0.0004242975954508257205094611511914\\
    &\approx 0.0424\%
\end{align*}
\end{enumerate}

\end{document}
